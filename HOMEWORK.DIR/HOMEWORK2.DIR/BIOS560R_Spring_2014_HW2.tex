\documentclass{article}
\usepackage{amsmath}
\usepackage{verbatim}
\usepackage{url}
\usepackage{pseudocode}

\usepackage{color}
\definecolor{darkblue}{rgb}{0,0,.75}
\usepackage{listings} %iclude code in your document

\lstloadlanguages{Matlab} %use listings with Matlab for Pseudocode
\lstnewenvironment{PseudoCode}[1][]
{\lstset{language=Matlab,basicstyle=\scriptsize, keywordstyle=\color{darkblue},xleftmargin=.04\textwidth,#1}}
{}

\usepackage{Sweave}
\begin{document}
\Sconcordance{concordance:BIOS560R_Spring_2014_HW2.tex:BIOS560R_Spring_2014_HW2.Rnw:%
1 15 1 1 0 280 1}

\title{BISO560R Spring 2014 Homework 2}
\author{Pittard}
\date{Due by 11:59 PM on February 14, 2014}
\maketitle


\section*{Instructions}
Send responses in a plain text file name LastName\_Firstname\_HW2.txt or LastName\_Firstname\_HW2.R. You can use RStudio to create the latter. We will run your commands at the R console to verify the statements. Email to BOTH \url{dvandom@emory.edu} and \url{wsp@emory.edu}

\section{Data Frames - 30 points}Write a function called my.sampler that fulfills the following requirements:
\\
It will take a data frame, such as mtcars, and ``split'' it up based on a given grouping variable. Then it will process each group and sample a specified number of records, (without replacement), from each group. When it is finished processing all groups it will combine those results into a data frame and return it. You need only accommodate a single group/factor per function call. As an example:

\begin{verbatim}
my.sampler(mtcars,mtcars$cyl,3)
                    mpg cyl  disp  hp drat    wt  qsec vs am gear carb
Toyota Corona      21.5   4 120.1  97 3.70 2.465 20.01  1  0    3    1
Honda Civic        30.4   4  75.7  52 4.93 1.615 18.52  1  1    4    2
Merc 230           22.8   4 140.8  95 3.92 3.150 22.90  1  0    4    2
Merc 280           19.2   6 167.6 123 3.92 3.440 18.30  1  0    4    4
Mazda RX4          21.0   6 160.0 110 3.90 2.620 16.46  0  1    4    4
Merc 280C          17.8   6 167.6 123 3.92 3.440 18.90  1  0    4    4
Hornet Sportabout  18.7   8 360.0 175 3.15 3.440 17.02  0  0    3    2
AMC Javelin        15.2   8 304.0 150 3.15 3.435 17.30  0  0    3    2
Cadillac Fleetwood 10.4   8 472.0 205 2.93 5.250 17.98  0  0    3    4

my.sampler(ChickWeight,ChickWeight$Diet,2)  # ChickWeight is built-in to R
    weight Time Chick Diet
20     138   14     2    1
57     197   16     5    1
253    145   16    23    2
270     49    2    25    2
361    221   16    32    3
389     41    0    35    3
513    135   12    45    4
554    322   21    48    4
\end{verbatim}

Your function will have to work with the other potentual grouping variables from the given data set (in the case of mtcars - gear, transmission,  carb, and vs ). Once you have split the data frame on a factor you should be able to, as part of your looping or apply logic, determine how many rows are in each category using functions you've learned about in class. We will also run your function on the ChickWeight data and the diamonds dataset, (in the ggplot2 package), to insure that there is nothing specific in it to mtcars
\\

THINGS TO CHECK: Insure that the number of unique values that the grouping variable takes on is less than 10. So if you are given a factor to split on then check that it assumes at most 10 unique values - otherwise complain to the user with an appropriate error message. 
\\

You should also do error checking to make sure that the number of records you are attempting to sample is not larger than the available number of records in the group. In that case then your program will stop with an informative message. For example in looking at the mtcars data, one can see that there is only one car that has a value of 6 in the carb column, thus
\begin{verbatim}
my.sampler(mtcars,mtcars$carb,2)  would fail:

Error in my.sampler(mtcars, mtcars$carb, 2) : 
  Number of requested samples 2 exceeds available records 

# As would the following because MPG really isn't a grouping variable:

my.sampler(mtcars,mtcars$mpg,5)
Error in my.sampler(mtcars, mtcars$mpg, 5) : 
  Grouping factor has 25 unique values. Must be less than 10
\end{verbatim}

\begin{table}[ht]
\caption{Arguments to my.sampler function}
\begin{tabular}{l | l}
\hline\hline
Argument & Purpose \\ [1ex]
\hline
my.df & a dataframe (e.g. mtcars, ChickWeight, etc) \\ [1ex]
\hline 
my.group & a grouping variable from my.df \\ [1ex]
\hline
numtosample & the number of records to sample from each value of my.group \\ [1ex]
\hline 
\end{tabular}
\label{table:nonlin}
\end{table}

\section{DNA - 30 points}

\subsection{seqgen - 10 points}Write a function called seqgen that returns a randomly generated DNA nucleotide sequence. Here we will restrict ourselves to using four characters - ACGT. For this exercise all 4 bases occur with the same probability. Your function should take the following arguments:

\begin{table}[ht]
\caption{Arguments to seqgen function}
\begin{tabular}{l | l}
\hline\hline
Argument & Purpose \\ [1ex]
\hline
size & the number of characters in the desired sequence \\ [1ex]
\hline 
collapsed (default TRUE) & return a collapsed character string or, if FALSE, a vector  \\ [1ex]
\hline
upper.case (default TRUE) & return upper or lower case \\ [1ex]
\end{tabular}
\label{table:nonlin}
\end{table}

As an example, a collapsed string would look like  "AGTTGG" whereas an uncollapsed string would look like ``A" ``G" ``T" ``T" ``G" ``G". Your function should validate the size of the requested sequence to make sure it makes sense. That is check for negative numbers or character strings. The second and third arguments should have defaults of TRUE and TRUE respectively. Here are some examples of how the function might be called:

\begin{verbatim}
seqgen(20,T,F)
[1] "ggtatacaatccccgtgggt"

seqgen(20,T,T)
[1] "GCTGCAATCTCGGCCTTGAT"

seqgen(20,F,T)
 [1] "G" "T" "A" "A" "G" "C" "A" "T" "C" "A" "G" "A" "A" "C" "G" "C" "T" "C" "C" "C"

seqgen(0)
Error in seqgen(0) :   Sorry - size needs to be numeric and greater than or equal to 1

\end{verbatim}

\subsection{revcomp - 20 points} 

Write a companion function called revcomp that, given any valid sequence generated by seqgen, will return the complimentary sequence. \emph{Note that for this assignment use of the ``chartr" function is forbidden}. In DNA terminology certain bases have specific compliments. The character A changes to T, C changes to G, G changes to C, and T changes to A. As an example the string \texttt{ACGTTGA} would have a compliment of: \texttt{TGCAACT}. 
\\

Moreover, if we wanted to take the ``reverse" compliment of a string we would first reverse the given string and then apply the compliment substitution as indicated. So given the same string as above, ``ACGTTGA", we would first reverse it to get ``AGTTGCA" after which we would apply the compliment substitutions to get a reverse compliment of ``TCAACGT". Use \url{http://www.bioinformatics.org/sms/rev_comp.html} to check your work. Make sure to use the drop down menu to reflect whatever it is you are testing: ``complement", ``reverse compliment", etc. Your function will take the following arguments:

\begin{table}[ht]
\caption{Arguments to revcomp function}
\begin{tabular}{l | l}
\hline\hline
Argument & Purpose \\ [1ex]
\hline
sequence & A sequence string like that returned by seqgen \\ [1ex]
\hline 
reverse (default FALSE) & If TRUE then return the compliment of the reverse DNA string \\ 
\hline
collapsed (default FALSE) & If TRUE then return a collapsed string \\
\hline
\end{tabular}
\label{table:nonlin}
\end{table}

\noindent
Here are some examples of how revcomp might be called:
\begin{verbatim}
my.seq = seqgen(20)

my.seq
 [1] "C" "A" "T" "T" "A" "T" "C" "A" "T" "C" "G" "C" "A" "A" "T" "T" "C" "T" "T"
[20] "A"

revcomp(my.seq)
 [1] "g" "t" "a" "a" "t" "a" "g" "t" "a" "g" "c" "g" "t" "t" "a" "a" "g" "a" "a"
[20] "t"

revcomp(seqgen(20),collapsed=T)
[1] "ccactcctattgatattata"
\end{verbatim}
\noindent
If you want a specific test sequence then use the following. However there should be nothing in your code that is specific to this string:
\begin{verbatim}

TAACGGCGGCAGGCGTATGG

revcomp("TAACGGCGGCAGGCGTATGG", reverse= F, collapsed=T )
[1] "attgccgccgtccgcatacc"

revcomp("TAACGGCGGCAGGCGTATGG", reverse=T, collapsed=T)
[1] "ccatacgcctgccgccgtta"

\end{verbatim}

\section{Quartile - 40 points}
Write a function called ``quartile.nist" which computes the requested percentiles of a given vector x. The calling sequence should look like this:
\begin{verbatim}
quartile.nist(x, probs=c(0.25,0.50,0.75))
\end{verbatim}

\begin{table}[ht]
\caption{Arguments to quartile.nist function}
\begin{tabular}{l | l}
\hline\hline
Argument & Purpose \\ [1ex]
\hline
x & a numeric vector of any size \\ [1ex]
\hline 
probs & a vector of desired percentile(s) where ( 0 < probs < 1 ) \\ 
\hline
\end{tabular}
\label{table:nonlin}
\end{table}

The default value for probs should be c(.25,.50,.75)). Check for values in probs that are less than or equal to zero or greater than or equal to 1. For hints on how to structure this function see the psuedocode given below in Figure 1.  Here are some examples of how this function can be used:
\begin{verbatim}

x = c(19,15.5,15.0,20.5,18.3,20.9,11.7,24.3,23.9)
quartile.nist(x)

 25%  50%  75% 
15.3 19.0 22.4

quartile.nist(x,c(.40,.60))
  40%  60% 
18.3 20.5 

set.seed(145)
testx = rnorm(1000)
quartile.nist(testx)
    25%     50%     75% 
-0.5810  0.0756  0.7543

\end{verbatim}

\subsection{Background}
Suppose that we have a data vector $x_1$, $x_2$, . . . , $x_n$. The pth percentile is conceptually meant to be a value $Q_p$ such that p proportion of the observations fall below $Q_p$. There are different ways to compute the percentiles / quartiles. Here we will use the ``n+1" method. Consider the data set:
\begin{verbatim}
Xn = 19.0 15.5 15.0 20.5 18.3 20.9 11.7 24.3 23.9  

x = c(19,15.5,15.0,20.5,18.3,20.9,11.7,24.3,23.9)

Here n = 9. We will first need to order these values:

sortedx = 11.7 15.0 15.5 18.3 19.0 20.5 20.9 23.9 24.3

\end{verbatim}
\noindent
To compute the 20th  percentile of the data we calculate $Q_{20}$ as follows:
\begin{verbatim}
(n + 1) * p = (9 + 1) *.20 = 2
\end{verbatim}
\noindent
So the value representing the 20th percentile is sortedx[2] which is 15.
Thus for computed Q values that wind up being an integer then you simply use the Q value to index into the sorted vector and you are done. Now, let's compute the 33rd percentile. Here we must use interpolation which will require some adjustment to our approach.
\begin{verbatim}
Q = (n + 1)*p = (9 + 1)*0.33 = 3.3 
\end{verbatim}
\noindent
Note that we have a number with a fractional part. HINT: It will be very useful for you to code up some logic to break up this number into its whole component (3) and its fractional component (0.3). You will need these values early on in your function to do comparisons. So here we take the third observation (the whole part of Q) of the SORTED vector and add to it 0.3 (the fractional part of Q) times the distance between it and the next highest (4th) observation. So:
\begin{verbatim}
Q33 = sortedx[3] + 0.3*(sortedx[4]-sortedx[3]) is 16.3

quartile.nist(x,.33)
   33% 
16.34 
\end{verbatim}
\noindent
So that is the approach you implement for computed Q values that have a fractional part. Well almost. There are ``boundary conditions" that we must consider. For low or high values of prob (like less than .10 or greater than .90) you need to pay attention to the computed Q value. Consider the case where probs is 0.95. Q = (9 + 1)*.95 = 9.5. Since there are only 9 elements in the original vector you can't interpolate it since there are no other elements above element 9. So in this case you would subtract 1 from the whole part (giving 8) and then apply the given interpolation formula:
\begin{verbatim}
Q95 = sortedx[8] + 0.5*(sortedx[9] - sortedx[8]) = 24.1 

quartile.nist(x,.95)
  95% 
24.1 

\end{verbatim}
\noindent
Also consider the case where probs is 0.05. Q = (9 + 1)*.05 = 0.5. Note that there is no \emph{zero-th} element of the vector. So we add 1 to Q and apply the given interpolation formula. So Q is 0.5 but we add 1 to it to get 1.5 and then apply the formula:
\begin{verbatim}
Q05 = sortedx[1] + 0.5*(sortedx[2] - sortedx[1]) = 13.3

quartile.nist(x,.05)
    5% 
13.35 
\end{verbatim}


\begin{figure}[h]
\caption{Possible Pseudocode for nist.quartile function}

\begin{PseudoCode}
determine the length, (n), of the input vector x 
sort the vector x for later use
setup an empty vector that will be used to return the requested quartiles

for each value in the probabilities vector do
    compute Q using the (n+1)*p formula
    split up Q into its whole and fractional parts
    if Q has a non zero fractional part
        implement some logic to check for low and hi boundary conditions;
        apply interpolation formula;
    else
        Q is a whole number with a zero fractional part;
        Use whole part of Q to index into the sorted array;
    end
    append the computed Q to the retrun vector
end
name the return vector elements to reflect the requested percentiles (e.g. 25%, 50%, etc)
\end{PseudoCode}
\end{figure}
\end{document}
