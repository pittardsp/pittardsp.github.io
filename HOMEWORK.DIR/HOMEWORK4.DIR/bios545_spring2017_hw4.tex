\documentclass[11pt]{article}
%\usepackage{graphicx, bbm, dsfont, color,float}
\usepackage{graphicx, bbm, color, float}
\usepackage{amssymb,amsfonts,amsmath}
\usepackage{geometry, hyperref}
\parskip=.1cm
\renewcommand{\baselinestretch}{1.2}

\geometry{margin=1in}
\begin{document}
\begin{center}
{\Large \bf Introduction to R programming: Homework 4}\\
Due on: March 10, 2017  11:59PM
\end{center}

\vspace{0.2in}
\noindent{\large\bf Problem 1} (35 + 3 points) 

\noindent Submit the R package you built during the lab. I expect to receive 
a single file called {\tt coolpkg\_1.0.tar.gz} or {\tt coolpkg\_1.0.zip} (version number might vary on different systems). 
The package must:
\begin{enumerate}
\item Be installable and loadable in R. (15 points)
\item Contain a function called {\tt myRowSums} that works as intended. (10 points)
\item Include a function help file for the {\tt myRowSums} function, with title,
description, usage, value and an example. (10 points)
\item  {\bf Bonus 3 points}: Includes a package vignette written in Sweave/knitr.
\end{enumerate}


\vspace{0.2in}
\noindent{\large\bf Problem 2} (30 + 3 points) \\
Write a brief report addressing the following research question: \textbf{What is the cross-sectional relationship between age and physical activity in American children?} Use the {\tt NHANES} dataset from the statistical analysis lecture. Use the variable \textbf{cpm} as a measure of physical activity, and include children age 6 to 17 years in your analysis. Here are some more detailed instructions:

\begin{enumerate}
\item Write an Introduction section providing a short description of the data. No more than 5 sentences. (10 points) 
\item Create a boxplot showing physical activity vs. age (10 points).
\item Fit and interpret a linear regression model for physical activity vs. age. Limit your interpretation to no more than 5 sentences. (10 points)
\item {\bf Bonus 3 points}: Write the report in Sweave/knitr. 
\end{enumerate}

\vspace{0.2in}
\noindent{\large\bf Problem 3} (20 points) \\
Provide simulation codes to assess the coverage rate of a 95\% confidence interval,
under the same linear regression setting as in the simulation lecture.  

\vspace{0.2in}
\noindent{\large\bf Problem 4} (15 points) \\
{\bf Buffon's needle problem} is a question posed  in the 18th century by 
{\bf Georges-Louis Leclerc, Comte de Buffon}:
``{\it Given a needle of length $a$ and an infinite grid of parallel lines with common distance $d$ between them, 
what is the probability that a needle, tossed at the grid randomly, will cross one of the parallel lines?}".

The problem is solved. When $a < d$, the answer is $p=2a/d\pi$. This result can be used for estimating $\pi$:
$\pi = 2a/dp$. The probability $p$ can be obtained from experiment: you keep throwing the needle 
and count the percentage of times it crosses the line. The experiment, however, can be realized computationally
by a simulation. 

Implement Buffon's needle idea 
and present the estimate of $\pi$ (mean and standard deviation over many trials) 
using sample sizes 1000, 10000 and 100000.
Submit the simulation code, and a short description of the method.

\end{document}